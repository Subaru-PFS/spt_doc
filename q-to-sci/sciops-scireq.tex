\documentclass[a4paper,notitlepage]{article}
\usepackage{ssn-format}
\drafttrue
\title{Items to be discussed on science requirements on PFS software}
\author{Atsushi Shimono + Naoyuki Tamura}

\definecolor{ccols}{rgb}{0.7,0.25,0}
\definecolor{ccolm}{rgb}{0,1,0}
\definecolor{ccoll}{rgb}{0,0.25,0.5}
\newcommand{\cols}[1]{\textcolor{ccols}{#1}}
\newcommand{\colm}[1]{\textcolor{ccolm}{#1}}
\newcommand{\coll}[1]{\textcolor{ccoll}{#1}}

\begin{document}

\SSNID{00003}
\SSNREV{000}
\SSNCATEGORY{ALL}
\SSNChangeRecord{Rev.001 / First Release / 2013--11--16}
\SSNReference{(none)}
\SSNAttachment{(none)}
\SSNWritten{Atsushi Shimono + Naoyuki Tamura}

\ssnhead

\section{Abstract}
\label{sec:abs}

This document is to start discussion to collect requirements from the science 
team to the PFS software. 
Especially from the survey operation point of view, what each software 
component need to do and what kind of user interfaces are necessary 
are expected to be clarified. 

In Section \ref{sec:backbround}, current ideas and understandings 
of possible styles of PFS surveys at the instrument builders' sides are 
presented, 
and in Section \ref{sec:questions}, questions and future discussion items 
are listed so that replies from the science team to the instrument builders 
can be considered as pieces of the science requirements to the PFS software. 


\section{PFS Survey Basics}
\label{sec:background}

Some basic features of PFS surveys can be listed as follows: 
\begin{itemize}
  \item PFS survey will consist of four sub-surveys 
    having different requirements as exemplified 
\footnote{feature? requirement?}
    in Table.~\ref{tab:sciops-scireq-subsvy}. 
  \item Sub-surveys may share fields and/or exposures.
  \item More targets may exist than fiber \# per single field, 
    so each field may be exposed multiple times, 
    depending on sub-surveys and their goals.
  \item Since a large number of fibers are available and the number of targets 
    is probably even larger, processes of survey planning, execution, and 
    data reduction should be automated as much as possible. 
\end{itemize}

\begin{table}[htb]
\caption{Requirements per sub-survey (sample; not completed)}
\label{tab:sciops-scireq-subsvy}
\begin{center}
\begin{tabular}{c|c|c|c|c}
conditions & cosmology & GA-LR & GA-MR & galaxy-AGN \\
\hline
\hline
typical integration time & 15min & 120min & 180min & 180min \\
\hline
Sky fiber number per field & ? & ? & ? & ? \\
\hline
Calibration star number per field & ? & ? & ? & ? \\
\hline
 : & : & : & : & : \\
\end{tabular}
\end{center}
\end{table}

\subsection{Processing a survey}
\label{sec:background:survey}

title\footnote{Processing? Executing? }


\subsubsection{Planning}
\label{sec:background:survey:planning}

Having target lists as inputs from all sub-surveys, 
fields of which centers correspond to telescope pointings are defined 
and fibers are then allocated to targets in each field
(e.g. Figure.~\ref{fig:sciops-scireq-slide-svyexp}). 

Multiple fiber
allocations may be considered on a single field so that more targets can
be observed. {\it Some rules need to be applied in defining the fields
(i.e. field center \& position angle are chosen) and allocating fibers
to targets, in order to maximize survey observation efficiency.}  
In the fiber allocation, some key instrument features such as collisions of
neighboring fiber positioners and field distortion on the PFI focal
plane also need to be taken into account.


In Figure.~\ref{fig:sciops-scireq-slide-svyexp}, 
each survey (A, B, C) will supply lists of targets (catalogues), 
and some survey might use several lists (e.g. A1, A2, and A3 at left-most 
schematic). 
Using these lists of targets, exposure fields (a, b, c, ...) will be defined 
and targets will be assigned to each field (or more than one field if we 
dither or make overlaps). 
For each field, we will have 2396 circular fiber positioner patrol area 
with some overrapped area (right two schematics). 
We assume to have multiple targets (objects) within each patrol area, 
and software will assign targets into fiber positioners and exposures. 
Some targets requires multiple exposures to complete required exposure time, 
and multiple exposures will be assigned for these targets
(middle bottom table). 

\begin{figure}[htb]
  \begin{center}
    \includegraphics[width=.75\linewidth]{sciops-scireq-slide-svyexp.png}
  \end{center}
  \caption{Schematic view of procedure from list of catalogues to exposure 
    configurations}
  \label{fig:sciops-scireq-slide-svyexp}
\end{figure}


\subsubsection{Observation at the summit}
\label{sec:background:survey:summit}

From instrument operation point of view, observation (series of exposures) 
will be executed as a flow in Figure.~\ref{fig:sciops-scireq-slide-oneexp}. 
IR detector data will be continuously read from detector during exposure 
(up-lamp), and will be finalized to a single image per detector just after 
end of exposure. Also CCD read out and acquisition and guidance (A\&G) 
procedure will start just after end of exposure, in pallarel. 
Following A\&G, fiber configuration will begin during or after A\&G completed 
(depends on slew rate etc.), and auto guiding (AG) procedure will follow 
after A\&G completed. 
Once fiber configuration finished and confirmed (also for AG), next exposure 
will start. 
Following data readout, finalized IR image and CCD image will be transferred 
to Hilo for on-site quick data reduction, during next exposure. 
This data transfer is assumed not to finish before start of next exposure, 
on-site quick data reduction will be performed during next exposure in 
pallarel. 
Results from on-site quick data reduction will be transferred to survey 
planning software and also observation operators to check and consider of 
following exposures. 

In Figure.~\ref{fig:sciops-scireq-slide-oneexp}, the basic instrument
operation and data flow in PFS observation is described as a
flowchart. Once the previous exposure is completed, the telescope starts
slewing to the next target field. Using the Acquisition \& Guide (A\&G)
cameras, field acquisition is performed and auto guiding is then
initiated. In parallel to these, fiber configuration is executed, and a
new exposure is started. While CCD detectors are read out only once at
the end of an exposure, IR detector is continuously read out during an
exposure (so-called up-lamp) and outputs a single FITS image at the end
making use of all the data in the intermediate readouts. The FITS data
from the CCD and IR detectors are transferred to Hilo and on-site quick
data reduction process is applied, during the next exposure (see the
). Results from this on-site quick data reduction will be transferred to
survey planning \& tracking software for records so that the operators
at the summit can check and consider them for subsequent exposures. Some
more details about the quick and full data reduction pipelines are given
in \S~\ref{datareduction}.


\begin{figure}[htb]
  \begin{center}
    \includegraphics[width=.75\linewidth]{sciops-scireq-slide-oneexp.png}
  \end{center}
  \caption{Activities during observation as timeline}
  \label{fig:sciops-scireq-slide-oneexp}
\end{figure}

%\paragraph{Updating exposure configuration to fit with condition}
%
%Actual observation PA (of telescope; PA${}_{\text{tel}}$) 
%will be defined at the Summit just before each exposure, 
%since rotation angle of PFI from the center need to be restricted within $\pm$ 
%60 degree ([TBC]) to make fiber FRD small as possible. 
%We have rotation symmetry by 60 degree on focal plane including fiber 
%positioner and A\&G camera, we can rotate PFI by 60 degree to get the same 
%PA of fiber distribution (PA${}_{\text{sky}}$) and exposure configuration 
%(excluding individual difference among each fiber). 
%To adjust PA${}_{\text{tel}}$ and configuration of PFI by rotating integral 
%multiplication of 60 degree from PA${}_{\text{sky}}$, 
%we need to update fiber configuration and object positions of A\&G and AG 
%with using final settings of PA${}_{\text{tel}}$, 
%to meet with small shift per each fiber positioner. 
%
%Also ETS need to consider of these possible error areas for collision 
%detection among positioners during fiber configuration trial, 
%while defining exposure configuration at off-site. 
%

\paragraph{QA on-site}
\label{sec:background:survey:qa}

For exposed data quality assuarance on site, 
PFS survey requires to perform it after each exposure, 
such as measurement of signal to noise ratios on exposed data. 
PFS will have on-site quick data reduction pipeline as reduced features' 
version of off-site full data reduction. 
With on-site quick look, observation sequence will be updated within a day 
or a run, but items to be performed by on-site quick data reduction 
(or its volume as computing time) directly 
connects how fast we can get evaluations for data quality, and we need to keep 
its volume to be low as reasonable processing time. 

Since this pipeline will be executed at Subaru during observation run just 
after each exposure, trade study between processing time and required accuracy 
could be important
\footnote{With making computing resources larger, we might be possible shorten 
processing time without reducing procedure in some level..}. 

Current assumpsions are 
\begin{itemize}
  \item Host at Hilo, transfer data in parallel or right after submitted to Gen2
    \begin{itemize}
      \item IR (up-ramp) : just transfer finalized image not raw up-ramp, right 
         after end of exposure and image finalization 
      \item other : transfer data right after end of readout
    \end{itemize}
  \item Calibrate with pre-built data, but not with simultaneous 2D/1D 
    calibration data (including PSF/FRD)
  \item Process during next exposure, and return data quality metrics (well?) 
    before beginning of next configuration
    \begin{itemize}
      \item How we can reduce pipelines will be a trade off between quality 
        and time of data processing
    \end{itemize}
\end{itemize}


\subsubsection{Data reduction}

To perform quick data quality assuarance on site by semi-realtime, 
PFS will have two modes of data reduction pipeline: quick-look mode on site, 
and full mode at off site. 
Former on-site quick-look data reduction need to be completed during 
the next exposure of data acuisition, data reduction pipeline will be 
simplized from full pipeline, both on pipeline and data used for reduction. 
Latter full mode will be two, one to be used just after each night or run 
of observation to reduce exposed data and to integrate into already executed 
survey, and second for periodical batch analysis. But pipeline will be 
the same for two modes (TBC). 

For how data will be reduced, 
detailed (somehow) description for data reduction pipeline are described 
in attached document from DRP team, PFS data reduction will be processed 
roughly as follows (see also Figure.~\ref{fig:sciops-scireq-drp-slide}): 
\begin{itemize}
  \item 2D extraction of fibers to 1D spectra
  \item Calibration and sky (continuum) subtraction (note: for bright sky 
    emission lines, currently not sure for detail, e.g. 2D PSF deconvolution)
  \item Combine multiple exposures into one spectrum and measure scientific 
    values
\end{itemize}

\begin{figure}[htb]
  \begin{center}
    \includegraphics[width=.75\linewidth]{sciops-scireq-drp-slide.png}
  \end{center}
  \caption{Schematic process for PFS data reduction pipeline}
  \label{fig:sciops-scireq-drp-slide}
\end{figure}



\subsection{PFS Software}

To perform all required activities on PFS, PFS software is defined as 
four packages: 
\begin{description}
  \item[ICS] Instrument Control Software (OBCP at Subaru)
    \begin{itemize}
      \item Everything from observation operation (provided exposure configuration) to output raw data.
      \item Including control software directly connected to hardware
      \item Including software for engineering, heartbeat, etc.
    \end{itemize}
  \item[DRP] Data Reduction Pipeline
    \begin{itemize}
      \item Reducing and calibrating raw data for science
      \item On-site quick look DRP, reduced data archive
    \end{itemize}
  \item[ETS] Exposure Targeting Software
    \begin{itemize}
      \item From possible target lists in a single field to fiber configuration for a single exposure
      \item Handling coordinate transformation related on sky coordinate (within one field)
    \end{itemize}
  \item[SPT] Survey Planning and Tracking software
    \begin{itemize}
      \item From large survey target lists to series of exposures on multiple fields
      \item Tracking survey progress including data QA for every exposure (by off-site full-DRP)
    \end{itemize}
\end{description}

Entire survey are planned and executed as schematic flow chart, 
Figure.~\ref{fig:sciops-scireq-slide-softcoord}. 
Items within this schematic flow are 


\begin{figure}[htb]
  \begin{center}
    \includegraphics[width=.75\linewidth]{sciops-scireq-slide-softcoord.png}
  \end{center}
  \caption{Sequence of top level activities for entire survey and 
    their associated package}
  \label{fig:sciops-scireq-slide-softcoord}
\end{figure}


From data point of view, each package will handle data or table as colorized 
in Figure.~\ref{fig:sciops-scireq-slide-data}. 
Supplied objects by catalogues from science team will be mapped into 
exposure field lists. For each exposure field list, multiple exposure 
configuration will be created, and results or intermediate outputs 
will be attached to each configuration and each field list. 
For each exposure configuration, exposed data and reduced results will be 
attached after execution of exposure. 

\begin{figure}[htb]
  \begin{center}
    \includegraphics[width=.75\linewidth]{sciops-scireq-slide-data.png}
  \end{center}
  \caption{Relationship between supplied and exposed data from catalogue to 
    survey tracking data}
  \label{fig:sciops-scireq-slide-data}
\end{figure}



\section{Questions and Discussion items}

This part lists discussion items listed from instrument builders' point 
of view, and might not cover all possible discussion or requirement item. 
Additionals are quite welcome. 

Color of each question number indicates: 
\begin{enumerate}
  \item[\cols{a}] Short term questions, like a month
  \item[\colm{b}] Short term, but assuming iteration between science and software
  \item[\coll{c}] Mid term questions, like a half year
  \item[d] Others
\end{enumerate}

\subsection{Survey definitions}

\begin{enumerate}
  \item[\colm{a}] Complete summary table like Table.~\ref{tab:sciops-scireq-subsvy}.
\end{enumerate}

\subsection{Functional requirements for survey observation}

This section covers what software need to perform or consider, 
on both types of activities: 
\begin{itemize}
  \item while defining series of real exposures from list of science targets, 
    both survey planning activities before survey observations (or submitting 
    survery proposal, if possible), and survey coordination during the survey 
    observation. 
  \item while observation is on-going at on-site, while one night or one run 
    is on execution, such as updating a sequence from exposed data or by 
    conditions in time. 
\end{itemize}

\subsubsection{Defining fields, dividing targets into exposures}

To define exposures, we need to define fields (field center and PA), and 
assign targets to each field and to each fiber. 

\paragraph{Defining fields}

Outermost shape of PFS exposure field is hexagonal, and we can tile by 
honeycomb structure without (large) gap between fields. 

\begin{enumerate}
  \item[\colm{a}] Do we have overlap region among fields or not?
  \item[\colm{a'}] If we have overlap, how much --- half? 1/3?
  \item[\colm{b}] Use the same field center, or dither per every exposure
  \item[\colm{c}] Use the same PA of field, or rotate by smaller than 60 degree?
\end{enumerate}


\paragraph{Restriction among multiple exposures for a single target}

Assumed single exposure time is 15min (900sec), we need to divide longer 
required exposure time into multiple exposures, e.g. 120min into 8 exposures. 

\begin{enumerate}
  \item[\coll{a}] Do we need to finish all exposures within a certain period, like 
    one day, one observation run, or one semestor/year? Or no limitation? 
  \item[\colm{b}] Some survey requires software to stop exposure after certain 
    signal to noise ratio achieved. 
    If we expose the same field with the same fiber configuration
    continuously, we cannot stop by the next exposure after achieval. 
    How do we do -- prevent such continuous exposure (change every objects 
    at any time), take survey time efficiency by ignoring such small number of 
    targets.
\end{enumerate}


\paragraph{If we dither, rotate or overlap fields}

If field definition is static and almost without gap nor overrap, 
one target will be mapped into one fiber positioner, and we don't need to 
care of complex multiple -- multiple assignments. 
But if not, we might need to care of many things... 

\begin{enumerate}
  \item[\cols{a}] How many targets we can assume per one fiber positioner patrol 
    area?
  \item[b] If we have multiple targets within one fiber positioner patrol 
    area, set of targets per patrol area will differ field by field 
    (e.g. target A and B are in area 1 for configuration x, 
    A in 100 and B in 101 for configuration y). 
    Survey progress will be complexed and it will take more exposurs to 
    complete "minimum set", is it ok?
  \item[c] Do we need to optimize exposure configurations and fiber -- target 
    assignments over entire survey period or just a exposure by a exposure? 
\end{enumerate}


\subsubsection{How to perform target selection per one field or exposure}

For each field (or each single exposure), PFS survery coordination software 
need to select a list of targets to be exposed from large lists of targets. 
Targets could be classified into several types, such as scientific objects, 
calibration objects (stars), sky region. 

\paragraph{Selection criteria}

\begin{enumerate}
  \item[\coll{a}] Selection criteria: complex merit function, or index (priority, 
    category) within supplied catalogue?
    \begin{itemize}
      \item merit function : like "index(f(mag) * g(color), h(z))
      \item index : like "target a,b = 1st / target c,d = 2nd" supplied within 
        catalog
    \end{itemize}
  \item[\coll{b}] How to integrate surveys: within single merit function or predefined 
    fraction per survey?
    \begin{itemize}
      \item When one field or exposure is shared by multiple surveys, we need 
        to combine two indexes supplied by each survey.
      \item How to do? Integrate into single merit function, or threshold like 
        "Survey A index 1 need to be observed up to 80\%", etc.
    \end{itemize}
\end{enumerate}

\paragraph{Miscellaneous targets}

Definitions of criteria or requirements for miscellaneous targets will 
vary per each survey, and we need to have a table like 
Table.~\ref{tab:sciops-scireq-subsvy}. 

\begin{enumerate}
  \item[a] How many fibers need to be assigned to these targets, including 
    distribution over a field? 
  \item[b] Sky fiber positions checked using HSC image or other scheme?
  \item[c] Calibration star selection criteria: color? intensity? instrument deformation?
\end{enumerate}



\subsubsection{Survey data quality assuarance and feedback to survey coordination}

Details of pipeline will be next section --- requirements for DRP. 

\begin{enumerate}
  \item[a] How we mark an object to be "done" --- 
    combination of on-site and off-site. 
    e.g. marked as done by on-site, but un-marked again by off-site later.
  \item[b] Just use on-site quick data reduction only for on-site update 
    within a day or a run, or record and use at off-site reduction and 
    coordination?
  \item[c] How to escalate outputs from on-site quick data reduction to 
    (full) survey planning?
\end{enumerate}


\subsection{Functional requirements for producing science output (DRP)}

This section covers requirements on data reduction pipeline from raw data to 
reduced 1D spectrum and scientific values extracted from spectrum 
\footnote{Check later section for reduced data and scientific values' archive.}.
Science requirements including reduction and calibration accuracies shall 
be discussed in separated for two modes. 

\subsubsection{on-site quick data reduction pipeline}

\begin{enumerate}
  \item[a] Time frame for on-site data reduction: 
    how "well" before beginning of next configuration?
  \item[b] Do we need different data quality measurement per sub-survey? 
  \item[c] What data characterization will be required: 
    noise level, peak or line signal to noise ratio, etc.?
    (to update a survey plan within a day)
\end{enumerate}


\subsubsection{off-site full data reduction pipeline}


For functional requirements, final output from data reduction pipeline will 
be survey archive data, such as table of astrophysical values, reduced data 
sets (1D spectra and required intermediate outputs). 

On this point, we need to define what information or output are required to 
be produced: 
\begin{enumerate}
  \item[a] Will all survey programs use the same reduction code, after 1D spectrum?
    (need to be different from program to program?)
  \item[b] What intermediate processing data are required, such as 1D with sky lines?
  \item[c] What to do for calibration targets (star, sky, etc.)?
\end{enumerate}

Also refer attached slide for overview of current DRP design to this document. 


\subsection{User interface requirements for survey coordination}

User Interface could be defined as stored data and their visualization. And 
requiremnts on the User Interface could be divided into two: 
"What data we need to store", and "What panel we need".
At this moment, system design is at conceptural level and no detailed design 
is planned, discussion of user interface is just on a side of stored data, but 
not actual panel list nor its mock-ups. 

From SDSS, Michael Strauss gave us a presentation titled "SDSS survey design 
\& operation: Inputs for PFS". 

We will have three phases for survey organization, and requirements should be 
specified per each phase: 
\begin{enumerate}
  \item[1] Pure organization / coordination phase (before starting survey)
  \item[2] Survey operation itself at the summit + pre-/post- observation works
  \item[3] Monitoring of survey progress
\end{enumerate}
Among these three phases, "Survey operation" will be performed mainly by small 
groups (of e.g. PDs), as currently in operation at Subaru. In the other two 
phases (and pre-/post- observation works?), many project members will attend 
or check.



\subsubsection{Pure organization / coordination phase (before starting survey)}

This phase is from lists of targets to fields and exposures, 
and some parts will be done by ETS but not by SPT (esp. per field). 

\begin{enumerate}
  \item[a] SPT will store supplied lists of objects (catalogues), what kind 
    of data will be supplied? 
    \begin{itemize}
      \item pre-defined table type? flexible xmldb type?
      \item data will also be used for merit function and object selection 
        (both by SPT and ETS).
    \end{itemize}
  \item[b] What input to and output from ETS shall be stored?
    \begin{itemize}
      \item Decision tables? Results from merit function? Or any other 
        intermediate products or decision data?
      \item Pre-input data used within SPT? - relates what algorithm we will use.
      \item To ETS, all output from ETS will be saved by SPT, what ETS need 
        to output? (section before)
    \end{itemize}
  \item[c] How we do for history of planning?
    \begin{enumerate}
      \item[c1] We need at least to save snapshots?
      \item[c2] How do we deal with version of catalog itself? Like using master 
        object ID, among all supplied object catalogue?
    \end{enumerate}
\end{enumerate}

\subsubsection{Survey operation itself at the summit + pre-/post- observation works}

Per night operation will be similar to classical typed observation
(by means of checking possible target fields (all, next), 
and statistics from exposed frames (or on-site DRP) will be used for 
operation. 

\paragraph{For operation}

\begin{enumerate}
  \item[a] What information shall be presented to operator to understand and 
    decide what operator need to and can do per night (or for next exposure)?
  \item[b] How level do we need to operate survey observation automatically?
    As for now, assuming somehow similar to classical typed observation..
\end{enumerate}

\paragraph{Recording telemetry and status}

\begin{enumerate}
  \item[a] What status and telemetry shall be transferred from ICS to SPT 
    related with exposure.
  \item[b] What statistics from on-site (quick) DR shall be recorded, all?
\end{enumerate}

\subsubsection{Monitoring of survey progress - ideas}

This covers checking progress status per survey, per field, per object: 
time, S/N etc. Could be one additional layer to survey archive??

\begin{enumerate}
  \item[a] What statistics need to be archived from off-site full-suite DR?
    \begin{itemize}
      \item per frame, per spectra, per field, per object
      \item into FITS header of processed image or survey record database?
    \end{itemize}
  \item[b] STP need to keep information in hierarchical to be displayed as layered UI.
  \item[c] How about progress history? -- we will integrate exposures per object
  \item[d] Searchable table : this will be UI design, but capable to be searchable 
    by many indexes??
\end{enumerate}


\subsubsection{Interface to ICS}

Many information required by SPT will be generated by other packages, we need 
to define what we need to keep for survey planning and tracking, and to 
allocate these requirements to other packages. 

Based on design concept of "keeping all information as much as possible", 
ICS will keep all telemetry (interval TBD) and logs (e.g. command, status), 
to use for defect detection and instrument performance check on the 
instrument control (e.g. performance of COBRA, A\&G/AG star).
On interface between SPT and ICS, what is required for SPT? Such as, instrument 
performance (e.g. fiber positioning error, A\&G/AG performance), instrument 
error report. Of course, we could get all archived data from ICS to SPT. 



\subsection{For survey archive}

Survey archive requirement and design is not well developed.

\begin{itemize}
  \item SDSS DR web as a starting point for a set of requiremnt?
  \item Automatic release of status/progress report per night or per run
    (As Gen2 of Subaru is doing: release PDF after each night)
  \item Visualization via kml? 
    (<kml xmlns="http://www.opengis.net/kml/2.2" hint="target=sky">)
\end{itemize}

Also, sample from LAM is attached to this document in separated. 

\subsection{For continuous data release}

As a background of survey archive, we need to perform continuous data 
reduction and periodic re-reduction of full data set.
Single(?) data reduction after each observation (single day or single run) 
could be performed on-premise (or even using on-site DR facility at Hilo 
supplied by us, although we need to care about already exposed data set...). 
And this will not be an issue.
For continuous data release like a "DR" of SDSS, Robert Lupton suggested to 
go cloud. Total "raw" data set will be around 0.5PB (to 1PB), and even after 
reduction it will be a level of 1PB including intermediate data.
It could be possible.

\begin{itemize}
  \item Do we need such periodical data release?
  \item What do we change on each data release --- pipeline upgrade?
\end{itemize}


\subsection{Other requirements for calibration}

Any requirements on calibrations?

\begin{itemize}
  \item Statistics of calibration? --- to check stability in short and long 
    term
  \item Cross-check by calibrating calibration object by themselves within a 
    frame
\end{itemize}



\appendix

\section{Related L2 requirements allocated to software}

Currently, several statistical requirements are already included 
into PFS L2 requirements: 
\begin{itemize}
  \item REQ-SYS-679, sky subtraction to the accuracy of 0.5\% [TBC] or better 
    of sky
  \item REQ-SYS-896 (890), subtract diffuse background to [TBD] e- accuracy
  \item REQ-SYS-688, flat-field systematics in fibers and detectors to 0.5\% 
    [TBC] or better
  \item REQ-SYS-696, wavelength to 0.01 pixel [TBC] accuracy
  \item REQ-SYS-656, relative flax of individual object spectrum to 5\% [TBC]
    accuracy
\end{itemize}




\end{document}

